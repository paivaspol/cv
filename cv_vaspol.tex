% LaTeX resume using res.cls
\documentclass[zhemargin]{res}
%\usepackage{helvetica} % uses helvetica postscript font (download helvetica.sty)
%\usepackage{newcent}   % uses new century schoolbook postscript font
%\setlength{\textwidth}{1in} % set width of text portion
%\setlength{\textwidth}{4in} % set width of text portion
\usepackage{enumitem}
\usepackage{hyperref}
\usepackage{color}

\usepackage{scrextend}
\changefontsizes[11pt]{9pt}

\setlist[itemize]{itemsep=-3pt, topsep=-5pt, partopsep=0pt}

\usepackage{geometry}
\geometry{
    left=0.5in,
    right=2in,
    top=1in,
    bottom=1in
}

\begin{document}

% Center the name over the entire width of resume:
 \moveleft.5\hoffset\centerline{\huge\bf Vaspol Ruamviboonsuk}
 \smallskip
% Draw a horizontal line the whole width of resume:
 %\moveleft\hoffset\vbox{\hrule width\resumewidth height 1pt}\smallskip
% address begins here
% Again, the address lines must be centered over entire width of resume:
 % \moveleft.5\hoffset\centerline{Computer Science and Engineering}
 \moveleft.5\hoffset\centerline{Computer Science and Engineering, University of Michigan, Ann Arbor, MI 48109-2121}
 \moveleft.5\hoffset\centerline{Email: vaspol@umich.edu $\cdot$ Phone: (678)-800-5952}
 % \moveleft.5\hoffset\centerline{}
 % \moveleft.5\hoffset\centerline{\color{blue} {\url{https://vaspol.me}}}
 \moveleft\hoffset\vbox{\hrule width 7.4in height 1pt}\smallskip

\begin{resume}

\section{\small EDUCATION}
	Ph.D. in Computer Science and Engineering \hfill Expected 8, 2020\\
	\textbf{University of Michigan}, Ann Arbor, MI\\
	Advisor: Prof. Harsha V. Madhyastha \\
  \textbf{Thesis: TBD}

	M.S. in Computer Science and Engineering \hfill 2016\\
	\textbf{University of Michigan}, Ann Arbor, MI\\
	Advisor: Prof. Harsha V. Madhyastha

	B.S. with Distinction in Computer Science \hfill 2014\\
	\textbf{University of Washington}, Seattle, WA\\
  Advisor: Prof. Richard Ladner\\
  Thesis: DigiTaps: Eyes-free Number Entry Method with Minimal Voice Feedback

\section{\small PUBLICATIONS}
    \textbf{Vroom: Accelerating the Mobile Web with Server-Aided Dependency Resolution}\\
    \underline{Vaspol Ruamviboonsuk}, Ravi Netravali, Muhammed Uluyol, and Harsha V. Madhyastha\\
    ACM SIGCOMM 2017, Los Angeles, CA, August 2017 \textbf{(ANRP Award)}

    \textbf{Demonstration of the Myria big data management service}\\
    Daniel Halperin, Victor Teixeira de Almeida, Lee Lee Choo, Shumo Chu, Paraschos Koutris,
    Dominik Moritz, Jennifer Ortiz, \underline{Vaspol Ruamviboonsuk}, Jingjing Wang,
    Andrew Whitaker, Shengliang Xu, Magdalena Balazinska, Bill Howe, and Dan Suciu\\
    2014 ACM SIGMOD international conference on Management of data, Snowbird, UT, November 2014

    \textbf{Tapulator: A non-visual calculator using natural prefix-free codes}\\
    \underline{Vaspol Ruamviboonsuk}, Shiri Azenkot, and Richard E Ladner\\
    Poster session, the 14th international ACM SIGACCESS conference on Computers and accessibility (ASSETS 2012), Boulder, CO, October 2012

\section{\small WORK EXPERIENCE}
    \textbf{Software Engineer Intern} \hfill 10/2018 - 3/2019 \\
    Google Inc. Ann Arbor, MI\\
    Team: Chrome Data Saver \\
    \emph{Project:} Wrap up the projects done with the team over Summer of 2017
    and 2018 by submitting a paper to the NSDI 2020 conference.

    \textbf{Software Engineer Intern} \hfill 6/2018 - 8/2018 \\
    Google Inc. Seattle, WA\\
    Team: Chrome Data Saver \\
    \emph{Project:} Optimized Chrome Lite Page render performance by
    identifying that fetches of CSS slows down rendering.
    %
    Sped up the rendering of web pages by 50\% by inlining CSS to the HTML,
    which eliminates the penalty of fetching CSS resources over the network.
    %

    \textbf{Software Engineer Intern} \hfill 9/2017 - 12/2017 \\
    Google Inc. Mountain View, CA\\
    Team: Ads Quality \\
    \emph{Project:} Worked on understanding the implications of different web page resource prefetching strategies.

    \textbf{Software Engineer Intern} \hfill 5/2017 - 8/2017 \\
    Google Inc. Seattle, WA\\
    Team: Chrome Data Saver \\
    \emph{Project:} Prototyped of a server-side rendering system to improve the
    user experience when browsing the web on slow cellular networks using
    resource-constrained devices; press release:
    \url{https://blog.chromium.org/2019/03/chrome-lite-pages-for-faster-leaner.html}.

    \textbf{Graduate Student Research Assistant} \hfill 5/2015 - Present \\
    Advisor: Prof. Harsha V. Madhyastha.\\
    Electrical Engineering and Computer Science Department, University of Michigan, Ann Arbor, MI

    \textbf{Software Developer Engineer in Test Intern} \hfill 6/2013 - 9/2013 \\
    Microsoft. Redmond, WA\\
    Project: Extended Windows Intune test framework to support fuzz testing, developed test modules
    using the extended features, and incorporated the module as part of the weekly test suite.

    % \textbf{Software Engineer Intern} \hfill 6/2012 - 8/2012 \\
    % Cobalt. Seattle, WA\\
    % Project: Designed and developed an internal monitoring tool that periodically aggregates application server logs for checking system health.


\section{\small RESEARCH EXPERIENCE}
	\textbf{Improving mobile web performance with aid from web servers}\\
    \textit{Advisor: Prof. Harsha V. Madhyastha \hfill 10/2015 - Present}
	\begin{itemize}
    \item Mobile page loads are slow because of underutilization of network and computational
      resources; a client can fetch the resources on a page only after it
      discovers them by parsing and executing other resources on the page.

    \item Designed and implemented the Vroom framework, which decouples discovery of
      resources from parsing and execution by leveraging recent web optimization
      techniques such as HTTP/2 PUSH and Link preload headers. Vroom is able to improve
      the median page load time by 5 seconds. [SIGCOMM'17]
	\end{itemize}

	\textbf{Improving latency from clients to cloud services}\\
    \textit{Advisor: Prof. Harsha V. Madhyastha \hfill 05/2015 - 10/2015}
	\begin{itemize}
    \item Analyzed internet measurement data to identify degradations of
      latency between clients and web service front-ends.
    \item Designed and implemented an algorithm for dynamically varying
      which front-end a client is redirected to in order to minimize the
      user-perceived latencies.
	\end{itemize}

	\textbf{Numerical input gestures for visually-impaired people}\\
    \textit{Advisor: Prof. Richard Ladner \hfill 2011 - 2014}
	\begin{itemize}
    \item Designed special gestures for inputting numbers on smartphones
      by leveraging multi-touch input surface for blind smartphone users. [ASSETS'12]
	\end{itemize}

	\textbf{Myria, Big Data as a Service}\\
    \textit{Advisor: Prof. Magdalena Balazinska \hfill 2012 - 2014}
	\begin{itemize}
    \item Implemented a database operator in the system. [SIGMOD'14]
	\end{itemize}

\section{\small AWARDS}
  Internet Research Task Force, \textbf{Applied Networking Research Prize}, 2018 \\
  University of Michigan, \textbf{First-year Ph.D. Student Fellowship}, 2014 \\
  University of Washington, \textbf{Mary Gates Research Scholarship}, 2013 \\
  % \begin{itemize}[leftmargin=*, topsep=-10pt] %\itemsep -1pt
	% %\setlength{\itemindent}{-2em}
  %   \item 2013 Mary Gates Research Scholarship. University of Washington, Seattle, WA
	% \end{itemize}


\section{\small TEACHING EXPERIENCE}
	\textbf{Graduate Student Instructor} \hfill 1/2017 - 4/2017 \\
	Department of Computer Science and Engineering, University of Michigan, Ann Arbor, MI
	\begin{itemize}
      \item EECS 498: Introduction to Distributed Systems (Winter 2017)
	\end{itemize}
	\textbf{Teaching Assistant} \hfill 9/2012 - 3/2014 \\
	Department of Computer Science and Engineering, University of Washington, Seattle, WA
	\begin{itemize}
      \item CSE 344: Introduction to Data Management (Fall 2012, Winter 2013, Winter 2014)
	\end{itemize}

\section{\small SKILLS}
  Most of my work is done in C/C++, Go, Java, and Python. I also have some
  experience working with networking tools such as iptables and tcpdump. I am
  also familiar with interacting with Android smartphones via ADB.

\end{resume}
\end{document}
