% LaTeX resume using res.cls
\documentclass[zhemargin]{res}
%\usepackage{helvetica} % uses helvetica postscript font (download helvetica.sty)
%\usepackage{newcent}   % uses new century schoolbook postscript font 
%\setlength{\textwidth}{1in} % set width of text portion
%\setlength{\textwidth}{4in} % set width of text portion
\usepackage{enumitem}
\usepackage{hyperref}
\usepackage{color}

\usepackage{scrextend}
\changefontsizes[11pt]{9pt}

\setlist[itemize]{itemsep=-3pt, topsep=-5pt, partopsep=0pt}

\usepackage{geometry}
\geometry{
    left=0.5in,
    right=2in,
    top=1in,
    bottom=1in
}

%\topsep + \parskip [+ \partopsep]

\begin{document}

% Center the name over the entire width of resume:
 \moveleft.5\hoffset\centerline{\huge\bf Vaspol Ruamviboonsuk}
% Draw a horizontal line the whole width of resume:
 %\moveleft\hoffset\vbox{\hrule width\resumewidth height 1pt}\smallskip
 \moveleft\hoffset\vbox{\hrule width 7.4in height 1pt}\smallskip
% address begins here
% Again, the address lines must be centered over entire width of resume:
 \moveleft.5\hoffset\centerline{Computer Science and Engineering}
 \moveleft.5\hoffset\centerline{University of Michigan, Ann Arbor, MI 48109-2121}
 \moveleft.5\hoffset\centerline{Email: vaspol@umich.edu}
 \moveleft.5\hoffset\centerline{Phone: (678)-800-5952}
 \moveleft.5\hoffset\centerline{\color{blue} {\url{http://vaspol.me}}}

\begin{resume}
 
\section{\small INTERESTS}
  My primary research interest is in web performance. Currently, I focus
  on making the web faster while preserving the web security model. I am also interested
  in other aspects of web performance such as metrics for measuring web performance.
  In addition, my research interests also include distributed systems, 
  which I have also done some research on wide-area Internet measurements, 
  and distributed storage systems.

\section{\small EDUCATION}
	Ph.D. student in Computer Science and Engineering \hfill Expected 2019\\
	\textbf{University of Michigan}, Ann Arbor, MI\\
	Advisor: Prof. Harsha V. Madhyastha

	M.S. in Computer Science. \hfill 2016\\
	\textbf{University of Michigan}, Ann Arbor, MI\\
	Advisor: Prof. Harsha V. Madhyastha

	B.S. with Honors in Computer Science \hfill 2014\\
	\textbf{University of Washington}, Seattle, WA\\
  Advisor: Prof. Richard Ladner\\
  Thesis: DigiTaps: Eyes-free Number Entry Method with Minimal Voice Feedback

\section{\small PUBLICATIONS}
    \textbf{Demonstration of the Myria big data management service}\\
    Daniel Halperin, Victor Teixeira de Almeida, Lee Lee Choo, Shumo Chu, Paraschos Koutris, 
    Dominik Moritz, Jennifer Ortiz, \underline{Vaspol Ruamviboonsuk}, Jingjing Wang, 
    Andrew Whitaker, Shengliang Xu, Magdalena Balazinska, Bill Howe, and Dan Suciu\\
    2014 ACM SIGMOD international conference on Management of data, Snowbird, UT, November 2014

    \textbf{Tapulator: A non-visual calculator using natural prefix-free codes}\\
    \underline{Vaspol Ruamviboonsuk}, Shiri Azenkot, and Richard E Ladner\\
    Poster session, the 14th international ACM SIGACCESS conference on Computers and accessibility (ASSETS 2012), Boulder, CO, October 2012

\section{\small RESEARCH EXPERIENCE}
	\textbf{Improving mobile web performance via aids from web servers}\\
    \textit{Advisor: Prof. Harsha V. Madhyastha \hfill 10/2015 - Present}
	\begin{itemize}
    \item Understood the fundamental reasons why web pages are slow and use this 
      knowledge to design a system that leverages recent advances of web optimization 
      techniques such as HTTP/2 PUSH and Link preload headers to reduce 
      web page load time.
	\end{itemize}

	\textbf{Improving front-end latency to cloud services}\\
    \textit{Advisor: Prof. Harsha V. Madhyastha \hfill 05/2015 - 10/2015}
	\begin{itemize}
    \item Analyzed internet measurement data to see patterns of front-end latency 
      degradation and understood how to implement front-end server redirection for the client.
	\end{itemize}

	\textbf{Numerical input gestures for visually-impaired people}\\
    \textit{Advisor: Prof. Richard Ladner \hfill 2011 - 2014}
	\begin{itemize}
    \item Designed special gestures for inputting numbers on smartphones 
      by leveraging multi-touch input surface. [ASSETS'12]
	\end{itemize}

	\textbf{Myria, Big Data as a Service}\\
    \textit{Advisor: Prof. Magdalena Balazinska \hfill 2012 - 2014}
	\begin{itemize}
    \item Implemented a database operator in the system. [SIGMOD'14]
	\end{itemize}

\section{\small TEACHING EXPERIENCE}
	\textbf{Teaching Assistant} \hfill 9/2012 - 3/2014 \\
	Department of Computer Science and Engineering, University of Washington, Seattle, WA
	\begin{itemize}
      \item CSE 344: Introduction to Data Management (Fall 2012, Winter 2013, Winter 2014)
	\end{itemize}

\section{\small WORK EXPERIENCE}
    \textbf{Software Developer Engineer in Test Intern} \hfill 6/2013 - 9/2013 \\
    Microsoft. Redmond, WA\\
    Project: Extended Windows Intune test framework to support fuzz testing, developed test modules using the extended features, and incorporated the module as part of the weekly test suite.

    \textbf{Software Engineer Intern} \hfill 6/2012 - 8/2012 \\
    Cobalt. Seattle, WA\\
    Project: Designed and developed an internal monitoring tool that periodically aggregates application server logs for checking system health.

    \textbf{Graduate Student Research Assistant} \hfill 5/2015 - Present \\
    Advisor: Prof. Harsha V. Madhyastha.\\
    Electrical Engineering and Computer Science Department, University of Michigan, Ann Arbor, MI

\section{\small AWARDS}
	\begin{itemize}[leftmargin=*, topsep=-10pt] %\itemsep -1pt
	%\setlength{\itemindent}{-2em}
    \item 2013 Mary Gates Research Scholarship. University of Washington, Seattle, WA
	\end{itemize}

\section{\small SKILLS}
	Most of my work is done in C/C++, Java, and Python, but I am most familiar 
  with Java and Python. I also have some experience working with networking 
  tools such as iptables, and tcpdump. I am also familiar with interacting 
  with android smartphones via ADB.

\end{resume}
\end{document}
